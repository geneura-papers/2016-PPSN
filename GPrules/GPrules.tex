\documentclass[a4paper,10pt,twocolumn,preprint,3p]{elsarticle}

\usepackage[latin1]{inputenc}
\usepackage{amssymb}
\usepackage{graphicx}
\usepackage{amsmath}
\usepackage{url}

\journal{Expert Systems with Applications}

\begin{document}

\begin{frontmatter}

% first the title is needed
\title{A Novel BYOD Security Rules Extraction Technique Based on Genetic Programming}

\author[ugr]{Paloma De las Cuevas}
\ead{palomacd@ugr.es}
\author[isgt]{Zeineb Chelly}
\ead{zeinebchelly@yahoo.fr}
\author[ugr]{Pablo Garc\'{\i}a-S\'anchez}
\ead{pablogarcia@ugr.es}
\author[ugr]{J.J. Merelo}
\ead{jmerelo@geneura.ugr.es}

\address[ugr]{Department of Computer Architecture and Computer Technology, ETSIIT and CITIC \\
University of Granada, Granada, Spain. Tel: +34958241778. Fax: +34958248993}
\address[isgt]{Laboratoire de Recherche Op\'{e}retionnelle de D\'{e}cision et de Contr\^{o}le de Processus. \\
Institut Sup\'erieur de Gestion de Tunis, Tunisia.}


\begin{abstract}
The growth in the number of personal devices in terms of variety and computational
abilities has given birth to a 
% (recent) Paloma - I erased this because I don't think it's recent anymore
concept in the corporate world
known as ``Bring Your Own Device'' (BYOD).
Within this concept,
employees are allowed to bring and work with their personal devices at
offices. Despite the BYOD's significant advantages such as reducing
business cost and increasing work productivity, the access to internal networks by these personal devices, for which
enterprises have limitations in controlling, exposes the companies to
security threats such as leak of confidential data and access by
unauthorized users. To handle these kinds of threats,
 there
is a need for a means of detecting and controlling abnormal user
access by establishing a classification rules based policy. Thus, in
this paper, we propose a Genetic Programming (GP) based framework for
BYOD security. In this environment, GP is used as a promising approach
capable of performing an automatic discovery of novel and interesting
classification rules, with the additional feature of presenting the new rules in an easily understandable way.
% (seen as a novel BYOD security rules extraction technique for detecting abnormal access.) This sentence is kind of redundant with the previous one. Changed for a new one, including more advantages of the GP framework.
The simulation results over real data and a
comparison with the results achieved by other techniques confirm the
viability, effectiveness, and applicability of the GP approach to the
BYOD security context.
\end{abstract}


\begin{keyword}
%TODO: Keywords
\end{keyword}

\end{frontmatter}


\section{Introduction}
\label{sec:intro}
% BYOD % adaptation by companies
% BYOD at schools
% (Zaineb) I added these paragraphs about BYOD in companies and schools
The fast pace of new technology has led modern computing to undergo
several notable transitions in a short period of time. Modern computing has moved
over time to smaller, more reliable and faster high-tech devises such
as smartphones, laptops and tablets. The use of these technologies in
several forms is progressing, and has led to a recent form of use known as
the ``Bring Your Own Device'' (BYOD) concept or philosophy. 

The BYOD concept has been integrated in several environments, mainly in
academia and the corporate world. In this last environment, the BYOD
practice refers to allowing employees to use their
personal laptops, smartphones, tablets, and other mobile devices in
the workplace for work-related tasks. The use of BYOD in the corporate world has many
advantages \cite{singh2012byod}; among them we mention saving costs --
as the company allows itself to save money on high-priced devices that
it would normally be required to purchase for their employees --, and
increasing flexibility and worker productivity as employees will not
be asked to haul around multiple devices to satisfy both their personal and
work needs, having everything they need in one device anytime and
anywhere.  
% (Paloma) This sentence was extremely long, sorry that I had to change it :(
Other advantages are tied to the increase of worker satisfaction,  attracting the best applicants and the increase of engagement in the workplace and after hours.

BYOD in the corporate world has since gained traction in the education
sector with an increasing number of schools around the world choosing
to implement their own BYOD policies. In such environment, BYOD (also
called Bring Your Own Technology) refers to a technology model that
allows students to bring their own devices to school for learning in
the classroom \cite{sangani2013byod, song2014bring}. The adoption of BYOD in schools
is argued by the fact that technology plays a leading role in
pupils/students' everyday lives and should, therefore, be an integral
part of their learning. However, for most schools it is financially
unsustainable to provide every student with the most appropriate
up-to-date device. BYOD is therefore considered an attractive,
cost-effective alternative, provided that many students usually own
devices that are superior and more up-to-date than those available in
schools. BYOD at schools has several benefits as well such as
personalising learning experiences, encouraging students' independent
learning, and promoting anytime, anywhere learning opportunities. 

Nevertheless and whatever the BYOD environment is, the biggest worry and the main issue is all about privacy and security \cite{miller2012byod}.
Focusing on the corporate world and as previously highlighted, the BYOD paradigm has several advantages
as it plays a leading role in increasing the
companies benefits.  Despite of that this
paradigm calls for a crucial need for securing the BYOD context. It is
clear that the uncontrolled access to internal networks by the
personal owned devices, for which companies have limitations in
controlling due to privacy preservation \cite{miller2012byod}, exposes the companies to security risks such as data
leakage, improper decommissioning, phishing, surveillance, and many
others \cite{lennon2012changing}. These threats have become the
companies main security concern, and for them it is a challenge to
assure a compromise between pushing personal devices towards
professional use and coping with their own stringent and complex
security requirements. This trend is inevitable as enterprises are
faced with questions of whether and how to manage this situation \cite{thomson2012byod}; and
thus every department must be involved in establishing security
policies and procedures to minimize the company's risks. In the
corporate world environment, the Corporate Security Policies (CSPs),
defined by the company's Chief Security Officer (CSO), are the core at
identifying threats and building a set of security rules. These aim at protecting
company assets by defining permissions to be considered for every
different action to be performed inside or outside the company's work
space, and eventually coming from the employees personal devices.

The aim of this paper is going beyond this traditional and simple
decision making process where a novel  BYOD security rules extraction
technique is proposed. The main idea is to create a reliable rule set
which is able to cover every new situation that may be a threat;
allowing the system to go beyond the limited set of known pre-defined
rules. This is achieved via the use of Genetic Programming (GP) which
has been emerged as a promising approach to deal with the problem of
discovering novel and interesting knowledge and rules from large
amount of data. Our proposed GP framework dedicated
for the BYOD context is capable of performing an automatic discovery
of classification rules.

The rest of the paper is organized as follows. In Section \ref{sec:SotA} we give an overview of the advances in GP applied to rule evolution and its applications; then, Section \ref{sec:problem} depicts the problem this work tries to solve, describing the available dataset and the proposed solution. The experimental set-up, as well as the different set of experiments that have been carried out are described in Section \ref{sec:experiments}. Section \ref{sec:gp} shows the obtained results from the application of GP to security rules extraction and, finally, the conclusions of this work along with some suggestions about how to continue our research are given in Section \ref{sec:future}. 

\section{Related Work}
\label{sec:SotA}

% (Paloma) To be extended

Since BYOD started to appear in companies' day-to-day philosophy a lot of research has been done about the advantages and disadvantages of this approach, as well as how to properly implement it in order to respect privacy while trying to secure the resources. Scarfo differentiates in \cite{scarfo2012new} two main ways to deal with the BYOD paradigm: allowing the device to connect via desktop or application virtualisation, meaning enough control to avoid employee's devices monitorisation; or control via Mobile Device Management (MDM), which has to be legally agreed with the employee. Ali et al. expand Scarfo's study in \cite{ali2015analysis} so that they review both BYOD access control models and BYOD security models. The authors further distinguish between MDM and Kernel Modifications inside the security models, and conclude with the description of a proposed model which combines most of the reviewed solutions, i.e. MDM and Virtual Private Network (VPN) access together with an encrypted container for the accessed information depending on the level of restriction. However, from all the papers in \cite{ali2015analysis} claimed to enforce policies, only in \cite{rhee2013high} the authors actually describe how the policies are enforced by describing which data is monitored. In any case, none of the papers mention the application of GP to the enforcement of the security policies.

From the point of view of market products, a variety of them have been released to help companies speed up the process of adoption of the BYOD concept. In \cite{de2015corporate} there is a description of the market solutions that the main manufacturers have developed to such purpose. The level of security and privacy preservation offered by these applications is different depending on the solution.

However, and to the best of our knowledge, there is not a tool that helps CSOs in developing new security rules via GP.

In \cite{DeFalco2002257}, a system which discovers rules for the PROBEN1 databases via GP is described. From the six databases inside PROBEN1 and analised by these authors, none is related to security.
In \cite{Tsakonas2004195} the authors also extract rules with \textsc{IF \ldots THEN} structure through GP, for medical purposes. 

% With regard to the extracted data, which quality is necessary to ensure low rates of noise and accurate extracted knowledge, researchers have advanced the state of the art in the precision of the devices. For instance, in \cite{rios2015mobile}, ...

\section{Problem Description}
\label{sec:problem}

As previously highlighted, the main idea behind the corporate security policies which are defined by the CSO is to build a basic, fixed and well defined set of rules, in the form of \textsc{IF \ldots THEN} clauses,  by which the company system allows or denies access to the company assets. In this sense and while facing an attack from a BYOD system, the set of rules will be tested looking for a matching between the access' characteristics and the rules' premises (the IF variables part of all rules). If a matching is found then the decision can be made, by checking the conclusion part of the rule set (the THEN part), either by allowing or denying employees' access to non-confident or non-certified data for example. However, it is important to mention that the companies' security rule set defined by the CSO is based on known and previously recognized accesses and thus it cannot cover the whole possibly safe and risky search spaces. Therefore, there is an urgent need to create a more reliable rule set which is able to cover every new situation that may be a threat. Hence, allowing the system to go beyond the limited set of known pre-defined rules.

Our proposed solution is based on a novel GP framework dedicated for the BYOD context capable of performing an automatic and wider discovery of classification rules. More precisely, our GP based framework will, first, extract all the possible values of every attribute in the data at hand and then make the ``RuleGeneticAlgorithm'' evolving. Specifically, in this context, each individual is seen as a set of rules. The best individual is the set which rules covers the maximum patterns. The last step would be to present the rules to the CSO of the company and tune the algorithm according to the decision. The description of the used data and further explicit details about our proposed solution are given in what is next.

\subsection{Available Data}
\label{subsec:data}

Anonymised user data has been used to perform the experiments for this work. The set of data has been gathered from the trials that were performed during the development of an FP7 European Project, called MUSES \cite{DBLP:conf/sac/MoraCGZJEBAH14}. In these trials, a group of users tested an application meant for securing a BYOD environment. The application generates warnings when the users acts in a dangerous way. Technically, these warnings were triggered by a set of initial and pre-defined rules, so that when certain conditions were met in an ``event'', the corresponding action could be allowed - nothing happened - or denied where a warning appeared.

The dataset contains, thus, a set of these ``events'' from which a number of attributes have been extracted or are given by the application itself. Therefore, the attributes can be classified in different ways, and one of them is the following:
\begin{itemize}
  \item Attributes given by the ``event detector'' in the application: These attributes are related to the type of the event (action), its timestamp, or origin, among others.
  \item Attributes inferred from the information in the database: The information given by the aforementioned attributes, along with the rest of information already existing in the database, helps inferring other attibutes.
      % (Zaineb) This sentence is weird. Please clarify it Paloma. I did not get what you mean here.
      % (Paloma) I hope it's more clear now
       These are, for instance: all information related to the origin, like the user position in the company or the device Operating System; the configuration of the device, such as WiFi or Bluetooth being enabled; and even lexical properties of the user password, in order to avoid storing the password itself or using it for classification or rule generation.
\end{itemize}

The trials had a duration of a month plus a week, and a total of 153270 events were registered in the database. It is important to note that from all of those events, almost 65\% were not useful for knowledge extraction purposes, as they were events of ``log in'', ``log out'', or ``restart''. The other 35\% were considered as \textit{important} because they contained information about user actions such as opening files or sending emails in a certain connection environment, changing security properties, or installing apps. Altogether, there are 38 attributes plus the class, which can take two possible values: GRANTED or STRONGDENY.

With respect to the balance between the classes, the dataset is unbalanced with the following ratio: 45856 instances are labelled as GRANTED and 3350 are labelled as STRONGDENY.

\subsection{Proposed Solution}
\label{subsec:solution}
% (zaineb) I prefer adding a sentence here as an introduction. What do you think Paloma?
% (Paloma) I agree, but as Pablo didn't finish the section, I prefer to wait :)
The proposed method uses GP to create a tree to model the different rules. The generated tree
is a binary tree of expressions formed by two different types of nodes:

\begin{itemize}
\item {\em Variable}: It is a logical expression formed by a name, an operator and a value. It is the equivalent to a ``primitive'' in the field of GP. \\
    Examples:
   \begin{math}
     \left \{
   \begin{array}{l}
     \texttt{password\_length<5} \\
     or \\
      \texttt{event\_level=>COMPLEX\_EVENT}
   \end{array}
   \right .
   \end{math}
\item {\em Action}: It is a leaf of the   tree and therefore, a ``terminal'' state. Each decision is the result of applying the rule; so it is limited to two terms which are \texttt{GRANTED} or \texttt{STRONGDENY}.
\end{itemize}

The different variables that have been used are as follows:

\begin{itemize}
\item {\em Binary Variable}: %TODO
\item {\em Categorical Variable}:
\item {\em Numerical Variable}:

\end{itemize}

The tree is translated to a set of rules starting from the leafs to the root node passing each time by the ancestors. An example of a rule can be presented as follows:
%TODO Show this more fancy :P (Pablo)

\begin{verbatim}
device_has_accessibility=false AND
wifiEnabled=false AND password_length<5 AND
wifiConnected=false AND
event_level=>COMPLEX_EVENT AND
user_role=>Administration AND
device_is_rooted=true THEN=STRONGDENY
\end{verbatim}


\section{Experiments Description}
\label{sec:experiments}

Sub-tree crossover and 1-node mutation evolutionary operators have been used, following our previous works that have used these operators obtaining good results \cite{CITAR_AQUI_EVOSTAR14GPBOT}. In this case, the mutation randomly changes the complete variable of a node or mutate the complete value. Each configuration is executed 30 times, with a population of 32 individuals and a 2-tournament selector for a pool of 16 parents.


Table \ref{tab:parameters} summarizes all the parameters used.

\begin{table}
\begin{center}
\begin{tabular}{|c|c|}
\hline
{\em Parameter Name} & {\em Value} \\\hline
Population size & 32 \\\hline
Crossover type & Sub-tree crossover \\ \hline
Crossover rate & 0.5\\ \hline
Mutation  & 1-node mutation\\ \hline
Mutation step-size & 0.25 \\ \hline
Selection & 2-tournament \\ \hline
Replacement & Generational with elitism\\ \hline
Stop criterion & 50 generations \\ \hline
Maximum Tree Depth & 10 \\ \hline %PABLO: CONFIRM THIS!
Runs per configuration & 30 \\ \hline
\end{tabular}
\caption{Parameters used in the experiments.}
\label{tab:parameters}
\end{center}
\end{table}

The used framework is OSGiLiath, a service-oriented evolutionary framework \cite{Garcia13Service}. The generated tree is converted to...  All the source code used in this work is available under a LGPL V3 License in \url{http://www.osgiliath.org}.

%\subsection{Results from Classifiers}
%\label{subsec:classifiers}

%\subsection{Results from Association Algorithms}
%\label{subsec:association}

%\subsection{Results from Clustering}
%\label{subsec:clustering}

%\subsection{Results from Genetic Programming}
%\label{subsec:gp}

%\section{Discussion}
%\label{sec:discussion}

\section{Results from Genetic Programming application}
\label{sec:gp}

\section{Conclusions and Future Work}
\label{sec:future}

\section*{Acknowledgments.}

This work has been supported in part by TIN2014-56494-C4-3-P (Spanish
Ministry of Economy and Competitivity), PROY-PP2015-06 (Plan Propio
2015 UGR).

\bibliographystyle{elsarticle-num}
\bibliography{GPrules}

\end{document}
