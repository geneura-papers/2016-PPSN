\documentclass[a4paper,10pt,twocolumn,preprint,3p]{elsarticle}

\usepackage[latin1]{inputenc}
\usepackage{amssymb}
\usepackage{graphicx}
\usepackage{amsmath}
\usepackage{url}

\journal{Expert Systems with Applications}

\begin{document}

\begin{frontmatter}

% first the title is needed
\title{A Novel Technique for the Extraction of BYOD Security Rules Based on Genetic Programming}

\author[ugr]{Paloma De las Cuevas}
\ead{palomacd@ugr.es}
\author[isgt]{Zeineb Chelly}
\ead{zeinebchelly@yahoo.fr}
\author[ugr]{Pablo Garc\'{\i}a-S\'anchez}
\ead{pablogarcia@ugr.es}
\author[ugr]{J.J. Merelo}
\ead{jmerelo@geneura.ugr.es}

\address[ugr]{Department of Computer Architecture and Computer Technology, ETSIIT and CITIC \\
University of Granada, Granada, Spain. Tel: +34958241778. Fax: +34958248993}
\address[isgt]{LARODEC, Institut Sup\'erieur de Gestion de Tunis, Tunisia.}


\begin{abstract}
The growth in the number of personal devices in terms of variety and computational
abilities has given birth to a concept known as ``Bring Your Own Device'' (BYOD), which applied into the corporate world has led companies to create policies geared towards allowing and enabling those devices.
This allow 
employees to bring and use their personal devices at
the company premises, on virtual private networks or otherwise while
on company work. Despite the significant advantages of this policy
such as reducing overheads and increasing work productivity, the
access to internal network and the assets it holds by these personal
devices, whose corporate control is forcefully limited, exposes the companies to
security threats such as leak of confidential data and access by
unauthorized users. % All this intro should be shorter and more to the
                    % point - JJ
%I have made it _larger_, better to explain concepts involved. If
%someone is able to make it shorter, please do - JJ
To handle these kinds of threats, we propose a way of detecting and controlling abnormal user
access by establishing a policy based on classification rules. Our approach for the problem of BYOD security uses a Genetic Programming (GP) based framework; GP is used as a promising approach
capable of performing an automatic discovery of novel and interesting
threat classification rules, with the additional feature of presenting the new rules in an easily understandable way.
The simulation results over real data and a
comparison with the results achieved by other techniques confirm the
viability, effectiveness, and applicability of the GP approach to the
BYOD security context.
\end{abstract}


\begin{keyword}
%TODO: Keywords
% Zaineb : I am proposing these keywords for now
Bring Your Own Device, Security, Genetic Programming, Rules Extraction. 
\end{keyword}

\end{frontmatter}


\section{Introduction}
\label{sec:intro}

The fast pace of new technology has led modern computing to undergo
several outstanding transitions in a short period of time. Modern
computing has moved over time to smaller, more reliable and faster
high-tech devices such as smartphones, laptops and tablets. The use of
these technologies in several forms is progressing, and has led to the
form of use known as the ``Bring Your Own Device'' (BYOD) concept or
philosophy. Since its first appearance in research
\cite{ballagas2004byod} as a way to call the interaction between
people's devices and a public display, such as art or advertising, it
has become a very popular practise, integrated into companies
\cite{thomson2012byod} and even schools \cite{song2014bring}.  

In the corporate world, the BYOD
practice refers to allowing employees to use their
personal laptops, smartphones, tablets, and other mobile devices in
for work-related tasks, but not necessarily while being in the workplace This has many
advantages \cite{singh2012byod}; among them we mention saving costs --
as the company allows itself to save money on high-priced devices that
it would normally be required to purchase for their employees --, and
increasing flexibility and worker productivity as employees will not
be asked to haul around multiple devices to satisfy both their personal and
work needs, having everything they need in one device anytime and
anywhere. 
Other advantages are tied to the increase of worker satisfaction, attracting the best candidates, and the increase of engagement in the workplace and after hours \cite{singh2012byod}. % references too - JJ
% (Paloma) All advantages come from the same paper, so I've repeated
% the reference.
% good :-) - JJ

%These advantages have made BYOD policies gain traction in the education
%sector with an increasing number of schools around the world choosing
%to implement their own BYOD policies. % reference!!! - JJ
% Also I don't see the point. Where do you want to bring this argument to? - JJ
% Ok, maybe we talk too much about BYOD at schools when we use company
% BYOD data, but it's true that schools are environments in which
% policies should be enforced too, right? Or can be applied in another
% way... So, what can we do? To create a section at the end called
% "other applications for our framework", for example? Or just mention
% it, reducing what's written now, and that's all?
% I would go for just mentioning it, because issues are much
% different. What company "assets" are we protecting here? - JJ
%In such environment, BYOD (also called Bring Your Own Technology) refers to a technology model that
%allows students to bring their own devices to school for learning in
%the classroom \cite{sangani2013byod, song2014bring}. The adoption of BYOD in schools
%is supported by the fact that technology plays a leading role in
%pupils/students' everyday lives and should, therefore, be an integral
%part of their learning. However, for most schools it is financially
%unsustainable to provide every student with the most appropriate
%up-to-date device. BYOD is therefore considered an attractive,
%cost-effective alternative, provided that many students usually own
%devices that are superior and more up-to-date than those available in
%schools. BYOD at schools has several benefits as well such as
%personalising learning experiences, encouraging students' independent
%learning, and promoting anytime, anywhere learning opportunities. % and this is important to know because... - JJ

However, there exists a big disadvantage concerning security, as
potentially unsecured devices from unaware users might interact with
important assets. Therefore, the main issue is to obtain a high level
of security, while maintining user privacy \cite{miller2012byod}.% if
                                % the issues are the same, why mention
                                % different environments separately?
                                % Besides, what assets of the learning
                                % center are you actually accessing? -
                                % JJ 
% Is this paragraph related to BYOD in schools? - JJ
Focusing on the corporate world and as previously highlighted, %my point exactly. Now you say we don't care about schools - JJ
% the BYOD paradigm has several advantages
%as it plays a leading role in increasing the
%companies benefits.  Despite of that this
%paradigm calls for a crucial need for securing company assets in the BYOD context. It is
it is clear that the uncontrolled access to internal networks by the
personal devices, for which companies have limitations in
controlling due to privacy preservation \cite{miller2012byod}, exposes the companies to security risks such as data
leakage, improper decommissioning, phishing including spear phishing, surveillance, and many
others \cite{lennon2012changing}. These threats have become the
companies main security concern, and for them it is a challenge to
assure a compromise between pushing personal devices towards
professional use and coping with their own stringent and complex
security requirements. This trend is inevitable as enterprises are
faced with questions of whether and how to manage this situation \cite{thomson2012byod}; and
thus every department must be involved in establishing security
policies and procedures to minimize the company's risks.

To this end, the Corporate Security Policies (CSPs), % I think a
                                % reference that defines them or to
                                % expand knowledge would be convenient
                                % here - JJ
defined by the company's Chief Security Officer (CSO), are the core at
identifying threats and building a set of security rules. The description of these jobs include protecting
company assets by defining permissions to be considered for every
different action to be performed inside or outside the company's work
space, and eventually coming from the employees personal
devices. Nonetheless, CSOs build the set of CSPs based on their
expertise, and as such, they have the limitation of not knowing every
possible combination of events that might lead to a dangerous
situation. 

The aim of this paper is to propose a novel technique for extracting
BYOD security rules that aids the CSO in the definition and refinement
of a set of security rules, that in the end, classify an upcoming
event or user action as permitted or not permitted. % The first claim
                                % about this paper can't be this
                                % imprecise. Say exactly what you want
                                % to do, not "going behond". Say that
                                % you will be helping define and
                                % refine those rules automatically,
                                % instead of having the CSO imagining
                                % and implementing, by herself, all
                                % the needed rules - JJ 
% (Paloma) I agree, changed it, what do you think?

%% Looks much better now :-) - JJ
The main idea is to create a reliable rule set
which is able to cover every new situation that may be a threat;
allowing the system to go beyond the limited set of known pre-defined
rules. In order to have the space of possible policy rules be as wide
as possible, we will need a technique that explores the rule space
efficiently and with the least assumptions about rule structure. 
% in order to have the space of possible policy rules be as wide as possible, we will need a techinque that explores the rule space efficiently and with as little structure as possible, that is why we introduce...
% You should say something like this. You need to motivate and justify the use of GP, mentioning previous work on the subject and how you really discovered that it was not enough - JJ
 This is why we have decided to use Genetic Programming (GP) for dealing with the problem of
discovering novel, interesting knowledge and rules from large
amounts of data. Considered part of the so-called \emph{Evolutionary
  Algorithms} \cite{back1996evolutionary}, GP are optimization
techniques inspired by natural evolution. By this method, the
solutions to a problem are internally encoded as trees, which can be
seen as a decision tree classifier \cite{safavian1990survey}.  
Our proposed GP framework dedicated
for the BYOD context is capable of performing an automatic discovery
of classification rules.
% Once again, there is a huge imbalance in the space and energy
% devoted to BYOD and the one you devote to GP. Except if you are
% going to publish in GPEM, which includes GP in the title, you will
% probably need to say a little bit more about GP. Even if it's GP,
% there are so many different ways if doing it that you will probably
% need to be more precise, specififying whether you mean "classic" GP
% or any other way. 
% JJ Approves this. You should say why you are using this technique as
% opposed to others - JJ

% This intro is very soft and does not have any "hard" claim. It does
% not really say what is specifically the problem we are dealing
% with, what we want to do and how it is addressing that
% problem. There should be a clear argumental line, something like 
% BYOD is very popular, but it presents security problems when you mix
% possibly compromised or unsecured devices with protected
% assets. Protecting that asset is the turf of the CSO, who defines
% security policies for asset access based on her expertise. However,
% these policies can't properly address the new threats posed by BYOD
% devices. That is why we need an automatic way of automatically
% defining these policies using known user actions and the possible
% threats they will eventually bring, especially when the action of an
% user might not result in immediate security danger, but in a
% possible future danger.
% I don't know, something like that, with more references, but with a
% clear argumental line. Without this argumental line, all the
% structure of the paper falls down in pieces. You can't create a
% proper state of the art, do the proper tests, and so on. - JJ
% This is pretty much addressed, but maybe we should have another look
% at it - JJ

The rest of the paper is organized as follows. In Section \ref{sec:SotA} we give an overview of the advances in GP applied to rule evolution and its applications; then, Section \ref{sec:problem} depicts the problem this work tries to solve, describing the available dataset and the proposed solution. The experimental set-up, as well as the different set of experiments that have been carried out are described in Section \ref{sec:experiments}. Section \ref{sec:gp} shows the obtained results from the application of GP to security rules extraction and, finally, the conclusions of this work along with some suggestions about how to continue our research are given in Section \ref{sec:future}. 

\section{Related Work}
\label{sec:SotA}

% (Paloma) To be extended

Since BYOD policies started to appear in companies' day-to-day policies  a lot
of research has been done about the advantages and disadvantages of
this approach \cite{singh2012byod}, as well as about how to properly implement it in order to
respect privacy while trying to secure the resources \cite{scarfo2012new, ali2015analysis, de2015corporate}. % this is a
                                % completely information-free
                                % claim. It does not claim anything,
                                % does not inform of anything. Adding
                                % a reference to a review would help,
                                % but in any case it does not help in
                                % any way your argument to show how
                                % many people are interested. You have
                                % to frame the problem in this first
                                % sentence. What's the BYOD "problem"?
                                % How has people attempted to nail it?
                                % - JJ
                                % (Paloma) It is an introduction to
                                % the works that are going to be
                                % presented next... is it so bad?
                                % Anyway... references added.
% Looks good. Lays out issues (it makes sense, or not, balance
% privacy/security) and it's not your claim, since you've added
% references :-) - JJ
These issues can be approached in different ways; Scarfo
differentiates in \cite{scarfo2012new} two main ones: allowing the
device to connect via desktop or application virtualisation, meaning
enough control to avoid employee's devices monitorisation; or allow
the company to control
via Mobile Device Management (MDM), which has to be legally agreed
with the employee. %Mention disadvantages: the company has to provide
                   %a virtualization platform and pay for cloud
                   %resources, or the employee has to allow the
                   %company to control its own device, which
                   %eventually dispels two of the advantages of BYOD
                   %policies: low overhead cost and preservation of
                   %privacy. (something like that). 
 Ali et al. expand on Scarfo's study in
\cite{ali2015analysis}  reviewing both BYOD access control
 and security models. The authors further distinguish
between MDM and Kernel Modifications inside the security models, and
conclude with the description of a proposed model which combines most
of the reviewed solutions, i.e. MDM and Virtual Private Network (VPN)
access together with an encrypted container for the accessed
information depending on the level of restriction. However, from all
the papers in \cite{ali2015analysis} claimed to enforce policies, only
in \cite{rhee2013high} the authors actually describe how the policies
are enforced by describing which data is monitored. In any case, none
of the papers mention the application of GP to the enforcement of the
security policies, but other techniques such as the implementation of
blacklists to avoid the installation of forbidden applications, or
whitelists to allow only certain ones. These techniques can be useful
with respect to the implementation of already defined security
policies, but our method allows to discover new security policies,
which in the case of black and white lists, can evolve them to include
new and/or malicious applications. 
 %Pablo: If previous papers describe some computational methods I would add to the last paragraph some of the techniques used. "...security policies, but other techniques such as ... These techniques can be useful with respect to... but our method is awesome because...".
 % (Paloma) I've included more information, please ckeck :D

This research has been accompanied by innovation in products released
to help companies implement or streamline the process of adoption of the BYOD
concept. In \cite{de2015corporate} there is a description of the
market solutions that the main manufacturers have developed to such
purpose. The level of security and privacy preservation offered by
these applications is different depending on the solution. % A SOA
                                % section establishes the state of the
                                % art, it's not a catalogue. As well
                                % as above, here you would have to say
                                % what is the disadvantage of these
                                % approaches and why you are going to
                                % do it better than them. 

%Non sequitur. What do you mean here? Why are you talking about this
%here? Extracted data from what? Start with something like "Previous
%papers have commented on the issue of using quality data to extract
%rules from that that are general and accurate..." - JJ
With regard to the extracted data, whose quality is necessary to
ensure low rates of noise  % What is noise in this context? - JJ
and accurate extracted knowledge,
researchers have advanced the state of the art in the precision of the
devices. For instance, in \cite{rios2015mobile}, the authors focus on
the importance of having an accurate measure of the location of the
devices. Although their approach seems to encroach upon employees'
privacy, they implemented a mobile information system for BYOD
adaptation and tested it in both Android and iOS devices. The authors
concluded that the efficiency of their system, along with the
possibilities of the hardware in the devices, results in a location
error of less than 50 meters with a 95\% confidence level. These
promising results allow the companies to use the location of their
employees to apply certain security policies.   

To the best of our knowledge, there is no tool that helps CSOs in
developing new security rules via GP, % But this is a very hard
                                % claim. It should go to the
                                % introduction!!! - JJ
 even as this method has been
indeed applied to classification -- from which classification rules
can be obtained -- , as described by Espejo et al. in
\cite{espejo2010survey}. Their work theoretically supports our
decision of applying GP to obtain security rules in a BYOD
environment. Furthermore, the works they review have applications in
all fields but not exactly the one we focus on here. The authors
studied three papers which use GP for classification with
communications data, but mainly for intrusion detection and e-mail
spamming. Also, in \cite{DeFalco2002257}, a system which discovers
rules for the PROBEN1 databases via GP is described. As happened in
the survey of Espejo et al., from the six databases inside PROBEN1 and
analysed by these authors, none is related to security. In
\cite{Tsakonas2004195} the authors also extract rules with \textsc{IF
  \ldots THEN} structure through GP, although for medical purposes. 
%Pablo: link previous paragraph with next one. Also, I would not use "on the other hand" below as it is not clear which is the "first hand". For example: "... rules via GP, even as this method has been applied to classification..."
% (Paloma) Done, check please!

%Once again, this paragraph does not follow the previous one. Maybe
%move it one paragraph up, where you are talking about proper training
%data. 
The lack of BYOD-related security databases is a hindrance for the
proliferation of research in this field. As seen, there actually exist
a number of datasets related to malware, e-mail spamming, or URL
requests \footnote{http://www.secrepo.com/}, but none about the use
that people make of their smartphones. This situation clearly exists
due to privacy considerations. Miller et al. \cite{Miller201253}
firmly state that the issue of privacy is even more important than the
security issue, although it does not receive the same
attention. Furthermore, when the MDM solution is described in
\cite{ali2015analysis}, the authors mention the privacy concerns the
employees may have due to the processing that the company makes over
all their behavioural data. And even after anonymising the resulting
dataset, one can still, in certain cases, identify a user of the
company the data comes from by processing the data. If this principle
cannot be guaranteed, the data should not be released
\cite{boillat2014handbook}. 
% TODO Pablo: I would explain here again the issue adding "This is the
% reason we can not release the dataset used in the experimental
% section, because our data includes..." BUT if we release it (issue
% #20), we can say that we are awesome and cool researchers by doing
% it 
% (Paloma) Ok then I leave this as TODO once we decide if we can
% release it.
% I think there's no problem with releasing it, but maybe you should
% ask Anna. - JJ

%Pablo: move this paragraph after the market product analysis. So, the
%SOA outline should be: 1) BYOD is a thing 2) There are some stuff
%done (market product, extracted data...) 3) However, in previous
%works there are stuff that SHOULD be done (GP and releasing datasets)
%for some fancy reasons, and 4) In this paper we do this and that to
%solve it. 
% Missing is "there is some trouble on finding data", but if we are
% not releasing this, maybe we should just be quiet about this - JJ
% (Paloma) Re-organised, I think well, but check in case I didn't move it to where you suggested.

\section{Problem Description}
\label{sec:problem}

As previously highlighted, the main idea behind the corporate security policies, which are defined by the CSO, is to build a basic, fixed, and well defined set of rules, in the form of \textsc{IF \ldots THEN} clauses,  by which the company system allows or denies access to the company assets. In this sense and while facing an attack from a BYOD system, the set of rules will be tested looking for a matching between the access' characteristics and the rules' premises (the IF variables part of all rules). If a matching is found then the decision can be made, by checking the conclusion part of the rule set (the THEN part), either by allowing or denying employees' access to non-confident or non-certified data for example. However, it is important to mention that the companies' security rule set defined by the CSO is based on known and previously recognized accesses and thus it cannot cover the whole possibly safe and risky search spaces. Therefore, there is an urgent need to create a more reliable rule set which is able to cover every new situation that may be a threat. Hence, allowing the system to go beyond the limited set of known pre-defined rules.

Our proposed solution is based on a novel GP framework dedicated for the BYOD context capable of performing an automatic and wider discovery of classification rules. More precisely, our GP based framework will, first, extract all the possible values of every attribute in the data at hand and then make the ``RuleGeneticAlgorithm'' evolving. Specifically, in this context, each individual is seen as a set of rules. The best individual is the set which rules covers the maximum patterns. The last step would be to present the rules to the CSO of the company and tune the algorithm according to the decision. The description of the used data and further explicit details about our proposed solution are given in what is next.

\subsection{Available Data}
\label{subsec:data}

Anonymised user data has been used to perform the experiments for this work. The set of data has been gathered from the trials that were performed during the development of an FP7 European Project, called MUSES \cite{DBLP:conf/sac/MoraCGZJEBAH14}. In these trials, a group of users tested an application meant for securing a BYOD environment. The application generates warnings when the users acts in a dangerous way. Technically, these warnings were triggered by a set of initial and pre-defined rules, so that when certain conditions were met in an ``event'', the corresponding action could be allowed - nothing happened - or denied where a warning appeared.

The dataset contains, thus, a set of these ``events'' from which a number of attributes have been extracted or are given by the application itself. Therefore, the attributes can be classified in different ways, and one of them is the following:
\begin{itemize}
  \item Attributes given by the ``event detector'' in the application: These attributes are related to the type of the event (action), its timestamp, or origin, among others.
  \item Attributes inferred from the information in the database: The information given by the aforementioned attributes, along with the rest of information already existing in the database, helps inferring other attibutes.
      % (Zaineb) This sentence is weird. Please clarify it Paloma. I did not get what you mean here.
      % (Paloma) I hope it's more clear now
       These are, for instance: all information related to the origin, like the user position in the company or the device Operating System; the configuration of the device, such as WiFi or Bluetooth being enabled; and even lexical properties of the user password, in order to avoid storing the password itself or using it for classification or rule generation.
\end{itemize}

The trials had a duration of a month plus a week, and a total of 153270 events were registered in the database. It is important to note that from all of those events, almost 65\% were not useful for knowledge extraction purposes, as they were events of ``log in'', ``log out'', or ``restart''. The other 35\% were considered as \textit{important} because they contained information about user actions such as opening files or sending emails in a certain connection environment, changing security properties, or installing apps. Altogether, there are 38 attributes plus the class, which can take two possible values: GRANTED or STRONGDENY.

With respect to the balance between the classes, the dataset is unbalanced with the following ratio: 45856 instances are labelled as GRANTED and 3350 are labelled as STRONGDENY.

\subsection{Proposed Solution}
\label{subsec:solution}
% (zaineb) I prefer adding a sentence here as an introduction. What do you think Paloma?
% (Paloma) I agree, but as Pablo didn't finish the section, I prefer to wait :)
The proposed method uses GP to create a tree to model the different rules. The generated tree
is a binary tree of expressions formed by two different types of nodes:

\begin{itemize}
\item {\em Variable}: It is a logical expression formed by a name, an operator and a value. It is the equivalent to a ``primitive'' in the field of GP. \\
    Examples:
   \begin{math}
     \left \{
   \begin{array}{l}
     \texttt{password\_length<5} \\
     or \\
      \texttt{event\_level=>COMPLEX\_EVENT}
   \end{array}
   \right .
   \end{math}
\item {\em Action}: It is a leaf of the   tree and therefore, a ``terminal'' state. Each decision is the result of applying the rule; so it is limited to two terms which are \texttt{GRANTED} or \texttt{STRONGDENY}.
\end{itemize}

The different variables that have been used are as follows:

\begin{itemize}
\item {\em Binary Variable}: %TODO
\item {\em Categorical Variable}:
\item {\em Numerical Variable}:

\end{itemize}

The tree is translated to a set of rules starting from the leafs to the root node passing each time by the ancestors. An example of a rule can be presented as follows:
%TODO Show this more fancy :P (Pablo)

\begin{verbatim}
device_has_accessibility=false AND
wifiEnabled=false AND password_length<5 AND
wifiConnected=false AND
event_level=>COMPLEX_EVENT AND
user_role=>Administration AND
device_is_rooted=true THEN=STRONGDENY
\end{verbatim}


\section{Experiments Description}
\label{sec:experiments}

Sub-tree crossover and 1-node mutation evolutionary operators have been used, following our previous works that have used these operators obtaining good results \cite{EvoStar2014:GPBot}. In this case, the mutation randomly changes the complete variable of a node or mutate the complete value. Each configuration is executed 30 times, with a population of 32 individuals and a 2-tournament selector for a pool of 16 parents.


Table \ref{tab:parameters} summarizes all the parameters used.

\begin{table}
\begin{center}
\begin{tabular}{|c|c|}
\hline
{\em Parameter Name} & {\em Value} \\\hline
Population size & 32 \\\hline
Crossover type & Sub-tree crossover \\ \hline
Crossover rate & 0.5\\ \hline
Mutation  & 1-node mutation\\ \hline
Mutation step-size & 0.25 \\ \hline
Selection & 2-tournament \\ \hline
Replacement & Generational with elitism\\ \hline
Stop criterion & 50 generations \\ \hline
Maximum Tree Depth & 10 \\ \hline %PABLO: CONFIRM THIS!
Runs per configuration & 30 \\ \hline
\end{tabular}
\caption{Parameters used in the experiments.}
\label{tab:parameters}
\end{center}
\end{table}

The used framework is OSGiLiath, a service-oriented evolutionary
framework \cite{DBLP:journals/soco/Garcia-SanchezGCAG13}. The
generated tree is converted to...  All the source code used in this
work is available under a LGPL V3 License in
\url{http://www.osgiliath.org}. 

%\subsection{Results from Classifiers}
%\label{subsec:classifiers}

%\subsection{Results from Association Algorithms}
%\label{subsec:association}

%\subsection{Results from Clustering}
%\label{subsec:clustering}

%\subsection{Results from Genetic Programming}
%\label{subsec:gp}

%\section{Discussion}
%\label{sec:discussion}

\section{Results from Genetic Programming application}
\label{sec:gp}

\section{Conclusions and Future Work}
\label{sec:future}

\section*{Acknowledgments.}

This work has been supported in part by TIN2014-56494-C4-3-P (Spanish
Ministry of Economy and Competitivity), PROY-PP2015-06 (Plan Propio
2015 UGR).

\bibliographystyle{elsarticle-num}
\bibliography{GPrules,geneura}

\end{document}
