
%%%%%%%%%%%%%%%%%%%%%%% file typeinst.tex %%%%%%%%%%%%%%%%%%%%%%%%%
%
% This is the LaTeX source for the instructions to authors using
% the LaTeX document class 'llncs.cls' for contributions to
% the Lecture Notes in Computer Sciences series.
% http://www.springer.com/lncs       Springer Heidelberg 2006/05/04
%
% It may be used as a template for your own input - copy it
% to a new file with a new name and use it as the basis
% for your article.
%
% NB: the document class 'llncs' has its own and detailed documentation, see
% ftp://ftp.springer.de/data/pubftp/pub/tex/latex/llncs/latex2e/llncsdoc.pdf
%
%%%%%%%%%%%%%%%%%%%%%%%%%%%%%%%%%%%%%%%%%%%%%%%%%%%%%%%%%%%%%%%%%%%


\documentclass[runningheads,a4paper]{llncs}

\usepackage[latin1]{inputenc}
\usepackage{amssymb}
\setcounter{tocdepth}{3}
\usepackage{graphicx}

\usepackage{url}
\newcommand{\keywords}[1]{\par\addvspace\baselineskip
\noindent\keywordname\enspace\ignorespaces#1}

\begin{document}

\mainmatter  % start of an individual contribution

% first the title is needed
\title{A Novel BYOD Security Rules Extraction Technique Based on Genetic Programming}

% a short form should be given in case it is too long for the running head
%\titlerunning{Lecture Notes in Computer Science: Authors' Instructions}

% the name(s) of the author(s) follow(s) next
%
\author{Paloma de las Cuevas \inst{1}%
\and Zeineb Chelly \inst{2}\and Pablo Garc\'ia \inst{1}\and Juan-Juli\'an Merelo \inst{1}}
%
\authorrunning{A Novel BYOD Security Rules Extraction Technique Based on Genetic Programming}
% (feature abused for this document to repeat the title also on left hand pages)

% the affiliations are given next; don't give your e-mail address
% unless you accept that it will be published
\institute{Dept. of Computer Architecture and Technology, University
of Granada, Spain \and LARODEC, Institut Sup\'erieur de Gestion de Tunis, Tunisia.}

\maketitle


\begin{abstract}
The growth of personal devices in terms of variety and computational
abilities has given birth to a new concept in the corporate world
known as ``Bring Your Own Device'' (BYOD).
% (Paloma) I don't think it is "new" anymore.
Within this concept,
employees are allowed to bring and work with their personal devices at
offices. Despite the BYOD's significant advantages such as reducing
business cost and increasing work productivity; the uncontrolled
access to internal networks by the personal devices, for which
enterprises have limitations in controlling, exposes the companies to
security threats such as leak of confidential data and access by
unauthorized users. To handle these inadequate countermeasures,
% (Paloma) I will look for another word which fits better here than "countermeasures", as by this word we are calling the threats "inadequate responses" (?)
 there
is a need for a means of detecting and controlling abnormal user
access by establishing a classification rules based policy. Thus, in
this paper, we propose a Genetic Programming (GP) based framework for
BYOD security. In this environment, GP is used as a promising approach
capable of performing an automatic discovery of novel and interesting
classification rules; seen as a novel BYOD security rules extraction
technique for detecting abnormal access. The simulation results and a
comparison with the results achieved by other techniques confirm the
viability, effectiveness and applicability of the GP approach to the
BYOD security context.
\end{abstract}


\section{Introduction}
\label{sec:intro}

% BYOD % adaptation by companies
% BYOD at schools

\section{Related Work}
\label{sec:SotA}

\section{Problem Description}
\label{sec:problem}
As previously highlighted, the BYOD paradigm has several advantages in the corporate world as it plays a leading role in increasing the companies benefits. This is achieved by allowing employees working with their own personal devices in and out of corporate environments. The benefits of BYOD are several; despite of that this paradigm calls for a crucial need for securing the BYOD context. It is clear that the uncontrolled access to internal networks by the personal owned devices, for which companies have limitations in controlling, exposes the companies to security risks such as data leakage, improper decommissioning, phishing, surveillance and many others \cite{lennon2012changing}. These threats have become the companies security concerns and for companies it is a challenge to assure a compromise between pushing personal devices towards professional use and coping with their own stringent and complex security requirements. This trend is inevitable as enterprises are faced with questions of whether and how to manage this situation; and thus every department must be involved in establishing security policies and procedures to minimize the company's risks. In the corporate world environment, the Corporate Security Policies (CSPs), defined by the company's Chief Security Officer (CSO), are the core at identifying and building a set of security rules aiming at protecting company assets by defining permissions to be considered for every different action to be performed inside or outside the company's work space; using eventually the employees personal devices. The main idea behind these CSPs is to build a basic, fixed and well defined set of rules by which the company system allows or denies access to the company assets. In this sense and while facing a pending from a BYOD system, the set of rules will be tested looking for a matching between the access' characteristics and the rules' premises (the If variables parts). If a matching is found then the decision can be made either by allowing or denying employees' access to non-confident or non-certified data for example. Yet, it is important to mention that the rule set defined by the CSO, in companies, is based on known accesses and thus it cannot cover the whole possibly safe and risky search spaces.

The aim of this paper is going beyond this traditional and simple decision making process where a novel  BYOD security rules extraction technique is proposed. The main idea is to create a reliable rule set which is able to cover every new situation that may be a threat; allowing the system to go beyond the limited set of known pre-defined rules. This is achieved via the use of Genetic Programming (GP) which has been emerged as a promising approach to deal with the problem of discovering novel and interesting knowledge and rules from large amount of data. Our proposed genetic programming framework dedicated for the BYOD context is capable of performing an automatic discovery of classification rules. More precisely, the GP based framework will, first, extract all the possible values of every attribute in the data at hand and then make the ``ruleGeneticAlgorithm" evolving. Specifically, in this context, each individual is seen as a set of rules. The best individual is the set which rules covers the maximum patterns. The last step would be to present the rules to the CSO of the company and tune the algorithm according to the decision.

% We have to talk about the data!

%\subsection{Available Data}
%\label{subsec:data}

%\subsection{Proposed Solution}
%\label{subsec:solution}

\section{Experiments Description}
\label{sec:experiments}

%\subsection{Results from Classifiers}
%\label{subsec:classifiers}

%\subsection{Results from Association Algorithms}
%\label{subsec:association}

%\subsection{Results from Clustering}
%\label{subsec:clustering}

%\subsection{Results from Genetic Programming}
%\label{subsec:gp}

%\section{Discussion}
%\label{sec:discussion}

\section{Results from Genetic Programming application}
\label{sec:gp}

\section{Conclusions and Future Work}
\label{sec:future}

\section*{Acknowledgments.}

This work has been supported in part by TIN2014-56494-C4-3-P (Spanish
Ministry of Economy and Competitivity), PROY-PP2015-06 (Plan Propio
2015 UGR).

\bibliographystyle{splncs}
\bibliography{GPrules}

\end{document}
