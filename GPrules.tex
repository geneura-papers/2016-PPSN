
%%%%%%%%%%%%%%%%%%%%%%% file typeinst.tex %%%%%%%%%%%%%%%%%%%%%%%%%
%
% This is the LaTeX source for the instructions to authors using
% the LaTeX document class 'llncs.cls' for contributions to
% the Lecture Notes in Computer Sciences series.
% http://www.springer.com/lncs       Springer Heidelberg 2006/05/04
%
% It may be used as a template for your own input - copy it
% to a new file with a new name and use it as the basis
% for your article.
%
% NB: the document class 'llncs' has its own and detailed documentation, see
% ftp://ftp.springer.de/data/pubftp/pub/tex/latex/llncs/latex2e/llncsdoc.pdf
%
%%%%%%%%%%%%%%%%%%%%%%%%%%%%%%%%%%%%%%%%%%%%%%%%%%%%%%%%%%%%%%%%%%%


\documentclass[runningheads,a4paper]{llncs}

\usepackage[latin1]{inputenc}
\usepackage{amssymb}
\setcounter{tocdepth}{3}
\usepackage{graphicx}

\usepackage{url}
\urldef{\mailsa}\path|{alfred.hofmann, ursula.barth, ingrid.haas, frank.holzwarth,|
\urldef{\mailsb}\path|anna.kramer, leonie.kunz, christine.reiss, nicole.sator,|
\urldef{\mailsc}\path|erika.siebert-cole, peter.strasser, lncs}@springer.com|    
\newcommand{\keywords}[1]{\par\addvspace\baselineskip
\noindent\keywordname\enspace\ignorespaces#1}

\begin{document}

\mainmatter  % start of an individual contribution

% first the title is needed
\title{Applying genetic programming for extracting classification rules}

% a short form should be given in case it is too long for the running head
%\titlerunning{Lecture Notes in Computer Science: Authors' Instructions}

% the name(s) of the author(s) follow(s) next
%
% NB: Chinese authors should write their first names(s) in front of
% their surnames. This ensures that the names appear correctly in
% the running heads and the author index.
%
\author{Paloma de las Cuevas \inst{1}%
\and Zeineb Chelly \inst{2}\and Pablo Garc\'ia \inst{1}\and Juan Juli\'an Merelo \inst{1}}
%
\authorrunning{Applying genetic programming for extracting classification rules}
% (feature abused for this document to repeat the title also on left hand pages)

% the affiliations are given next; don't give your e-mail address
% unless you accept that it will be published
\institute{Dept. of Computer Architecture and Technology, University
of Granada, Spain \and Laboratoire de Recherche Op\'erationelle de D\'ecision et de Contr\^ole de Processus, Institut Sup\'erieur de Gestion, Tunisia.}

%
% NB: a more complex sample for affiliations and the mapping to the
% corresponding authors can be found in the file "llncs.dem"
% (search for the string "\mainmatter" where a contribution starts).
% "llncs.dem" accompanies the document class "llncs.cls".
%

%\toctitle{Lecture Notes in Computer Science}
%\tocauthor{Authors' Instructions}
\maketitle


\begin{abstract}
This is the abstract.
\end{abstract}


\section{Introduction}
\label{sec:intro}

\section{Related Work}
\label{sec:SotA}

\section{Problem Description}
\label{sec:problem}

\subsection{Available Data}
\label{subsec:data}

\subsection{Proposed Solution}
\label{subsec:solution}

\section{Experiments Description}
\label{sec:experiments}

\subsection{Results from Classifiers}
\label{subsec:classifiers}

\subsection{Results from Association Algorithms}
\label{subsec:association}

\subsection{Results from Clustering}
\label{subsec:clustering}

\subsection{Results from Genetic Programming}
\label{subsec:gp}

\section{Discussion}
\label{sec:discussion}

\section{Future Work}
\label{sec:future}

\section*{Acknowledgments.} The heading should be treated as a
subsubsection heading and should not be assigned a number.

\bibliographystyle{splncs}
\bibliography{GPrules}

\end{document}
